% !TEX encoding = UTF-8 Unicode
\section{Formulación y nomenclatura de Química Inorgánica}

\begin{prob}
Nombra los siguientes compuestos:
\begin{multicols}{3}
\begin{enumerate}
	\item \ce{HBr}
	\item \ce{H_3PO_4}
	\item \ce{HIO_3}
	\item \ce{HSO_4^-}
	\item \ce{AlCl_3}
	\item \ce{Al_2(SO_4)_3}
	\item \ce{SeO_2}
	\item \ce{Pb^{2+}}
	\item \ce{N_2O_4}
	\item \ce{BaO}
	\item \ce{HgO}
	\item \ce{Cd(OH)_2}
	\item \ce{Mn_2SiO_4}
	\item \ce{Ca(HSO_4)_2}
	\item \ce{CaCO_3}
	\item \ce{ZnSO_4 7H_2O}
	\item \ce{Na_2O_2}
	\item \ce{Co_3(PO_4)_2}
	\item \ce{ZnBr_2}
	\item \ce{KMnO_4}
	\item \ce{CuI_2}
	\item \ce{Ba(NO_3)_2}
	\item \ce{CuSO_4}
	\item \ce{CoCl_2}
	\item \ce{CrCl_3}
	\item \ce{Na_2S}
	\item \ce{PO_4^{3-}}
	\item \ce{HNO_3}
	\item \ce{HCl} (aq)
	\item \ce{K_2CrO_4}
\end{enumerate}
\end{multicols}
\end{prob}

\begin{prob}
Formula los siguientes compuestos:
\begin{multicols}{2}
\begin{enumerate}
	\item Fosfato de aluminio
	\item Bromuro de amonio
	\item Ácido cianhídrico
	\item Cianuro de potasio
	\item Ión sulfito
	\item Ión yoduro
	\item Óxido de aluminio
	\item Óxido de cobre (II)
	\item Carbonato de bario
	\item Cromato de plata
	\item Catión hierro (III)
	\item Sulfuro de disodio
	\item Nitrito de sodio
	\item Ión hidrogenosulfuro
	\item Dicromato de calcio
	\item Fosfato de sodio
	\item Fosfina
	\item Bicarbonato de sodio
	\item Peróxido de hidrógeno
	\item Tiosulfato de potasio 
\end{enumerate}
\end{multicols}
\end{prob}
