% !TEX encoding = UTF-8 Unicode
\section{Estequiometría y energía de las reacciones químicas}

\begin{prob}
Las bolsas de ''aire de seguridad'' (\textit{airbag}) de los automóviles se inflan con
nitrógeno gaseoso generado por la rápida descomposición de ácido de sodio (\ce{NaN3}):
$$ \ce{NaN3 (s) -> Na (s) + N2 (g)} $$
Si una bolsa de aire tiene un volumen de 38 L y debe llenarse con nitrógeno gaseoso a una
presión de 1'5 atm y una temperatura de $25ºC$, ¿cuántos gramos de ácido deben descomponerse?
\end{prob}

\begin{prob}
En siderurgia, la cal viva (\ce{CaO}) se combina con la sílice (\ce{SiO2}) presente en
el mineral de hierro para dar una escoria fundida de fórmula \ce{CaSiO3}.
\begin{enumerate}[a)]
	\item ¿Qué masa de escoria se obtiene a partir de una tonelada de sílice? ¿Qué masa 
	de cal viva es necesaria?
	\item La cal viva necesaria se puede obtener descomponiendo por calor la caliza 
	(\ce{CaCO3}) para dar cal y \ce{CO2}. ¿Qué masa de caliza haría falta?
\end{enumerate}
\end{prob}

\begin{prob}
Los camellos almacenan al grasa triestearina (\ce{C57H110O6}) en su giba. Además de
constituir una fuente de energía, la grasa es una fuente de agua, ya que se produce la
siguiente reacción:
$$ \ce{C57H110O6 (s) + O2 (g) -> CO2 (g) + H2O (l)} $$
¿Qué masa de agua puede obtenerse a partir de 1 kg de grasa?
\end{prob}

\begin{prob}
Escribe la reacción de combustión del pentano (\ce{C5H12}) sabiendo que se desprenden
3537 kJ / mol.
\begin{enumerate}[a)]
	\item ¿Qué cantidad de energía desprenderá una bombona de pentano de 10 kg?
	\item ¿Cuántos litros de aire (21\% de \ce{O2}) medidos en condiciones normales son 
	necesarios para su combustión?
\end{enumerate}
\end{prob}

\begin{prob}
Calcula el volumen de disolución 0'8M de ácido nítrico que reacciona con $50cm^3$ de una
disolución 2M de hidróxido de magnesio. En el proceso se obtienen nitrato de magnesio y
agua.
\end{prob}

\begin{prob}
El ácido acético, \ce{CH3COOH}, es el responsable de la acidez del vinagre. Se hacen
reaccionar 10'03 g de un vinagre comercial con hidróxido de bario $1'76 \cdot 10^{-2}M$,
siendo necesarios 137'20 mL de hidróxido para neutralizar el ácido. Calcula el porcentaje,
en masa, de ácido acético que contiene la muestra de vinagre.
\end{prob}

\begin{prob}
Se dispone de una muestra de cinc que se hace reaccionar con una disolución de ácido
clorhídrico de 1'18 g / mL de densidad y 35 \% de riqueza. Como producto de la reacción se
obtienen cloruro de cinc e hidrógeno gas. Calcula:
\begin{enumerate}[a)]
	\item La concentración molar del ácido clorhídrico.
	\item La masa de cinc de la muestra sabiendo que, para reaccionar con la misma, se
	necesitan $30cm^3$ de la disolución del ácido.
\end{enumerate}
\end{prob}

\begin{prob}
Se hacen reaccionar entre sí dos disoluciones acuosas que contienen 45 g de hidróxido de
bario y 18 g de ácido clorhídrico, respectivamente. En el proceso se forman cloruro de
bario y agua. Indica cuál es el reactivo limitante. Calcula la masa de cloruro de bario 
que se obtiene y la masa de reactivo que queda sin reaccionar.
\end{prob}

\begin{prob}
Se hacen reaccionar entre sí dos disoluciones acuosas que contienen 45 g de hidróxido de
bario y 18 g de ácido clorhídrico, respectivamente. En el proceso se forman cloruro de
bario y agua. Indica cuál es el reactivo limitante. Calcula la masa de cloruro de bario 
que se obtiene y la masa de reactivo que queda sin reaccionar.
\end{prob}
