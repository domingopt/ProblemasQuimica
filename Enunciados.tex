\documentclass[11pt,twoside,a4paper]{article}
\usepackage[spanish]{babel}
\usepackage{amsthm}
\usepackage[utf8]{inputenc}
\usepackage{geometry}                % See geometry.pdf to learn the layout options. There are lots.
\geometry{letterpaper}                   % ... or a4paper or a5paper or ... 
%\geometry{landscape}                % Activate for for rotated page geometry
%\usepackage[parfill]{parskip}    % Activate to begin paragraphs with an empty line rather than an indent
\usepackage{graphicx}
\usepackage[version=3]{mhchem}
\usepackage{amssymb}
\usepackage{epstopdf}

\DeclareGraphicsRule{.tif}{png}{.png}{`convert #1 `dirname #1`/`basename #1 .tif`.png}

\title{Problemas de Quimica}
\author{Maria Trillo Alcala}
%\date{}                                           % Activate to display a given date or no date

%definition of the new style
\newtheoremstyle{problem}% name 
  {8pt}% Space above 
  {12pt}% Space below 
  {}% Body font
  {}% Indent amount
  {\bfseries}% Theorem head font 
  {}% Punctuation after theorem head 
  {6pt}% Space after theorem head
  {}% Theorem head spec

%declaring the style to be used
\theoremstyle{problem}
%declaring the new theorem-like structure and declaring that
%it should be numbered within chapters
\newtheorem{prob}{Ejercicio}[section]

\begin{document}

\section{Incluir el tema 1}

\section{Propiedades de la Tabla Periódica}

\begin{prob}
Indica cuántos electrones externos tienen los siguientes átomos: Ca, B, N, K, I.
Para responder es preciso considerar el grupo y el periodo de la Tabla Periódica al que pertence
cada uno de ellos. 
\end{prob}


\begin{prob}
¿Qué elementos tienen mayor radio: el bario o el calcio? ¿Cuál es el más electronegativo el O o el Ar?
\end{prob}


\begin{prob}
Dados los elementos Ar (Z=8), As (Z=33) y I (Z=53). Indica el grupo y el periodo al que pertenecen.
¿Cuál de ellos tiene una mayor energía de ionización? 
\end{prob}


\begin{prob}
¿Qué elementos tienen más tendencia a formar iones positivos, el berilio o el bario?
¿Cuál tiene más tendencia a formar iones negativos, el carbono o el oxígeno?
\end{prob}


\section{El enlace químico}

\begin{prob}
Dibuja las estructuras de Lewis que cumplan la regla del octeto de:
\begin{enumerate}
	\item Anión carbonato
	\item Amoniaco
	\item Ácido clórico
\end{enumerate}
\end{prob}


\begin{prob}
Teniendo en cuenta la teoría de los enlaces de valencia, explica las valencias
covalentes del flúor, del oxígeno y del azufre. 
\end{prob}


\begin{prob}
Teniendo en cuenta la teoría anterior, explica y representa las siguientes moléculas:
$Cl_2$, $HCl$, $O_2$, $N_2$. ¿Cuáles presentan enlaces múltiples?
\end{prob}


\begin{prob}
¿Se precisa la misma energía para romper un enlace $\sigma$ que un $\pi$?\\
Según la T.E.V., ¿qué forma tendrá la molécula de agua? Clasifica sus enlaces covalentes
en función de su polaridad.
\end{prob}


\begin{prob}
Observa los puntos de ebullición de las siguientes sustancias. Justifica sus valores de acuerdo con
la naturaleza de sus enlaces y / o fuerzas intermoleculares.
\vspace{8pt}

\begin{tabular}{|c|c|}
\hline
Sustancia covalente& Punto de ebullición $(ºC)$\\
\hline
$O_2$&-183\\
$H_2O$&100\\
$I_2$&183\\
$C$ (diamante)&4200\\
\hline
\end{tabular}
\end{prob}


\begin{prob}
Indica la carga esperada para los iones que componen las siguientes sustancias basándote en el grupo
a que pertenecen y sus características como metales o no metales. Nombra a dichos iones.
\begin{itemize}
\item $NaCl$
\item $CaBr_2$
\item $MgS$
\item $BaO$
\item $Al_2O_3$
\item $SrH_2$
\item $KH$
\item $Na_2O_2$
\item $BnOH$
\item $NH_4F$
\end{itemize}
\end{prob}


\begin{prob}
Clasifique las siguientes sustancias según el tipo de enlace que presenten:
\begin{itemize}
\item $MgCl_2$
\item $PCl_5$
\item $Au$
\item $Fe$
\item $SO_2$
\item $H_2O$
\item $NH_3$
\end{itemize}
\end{prob}


\begin{prob}
Teniendo en cuenta los puntos de fusión y de ebullición de las siguientes sustancias que aparecen en la tabla. 
Justifica sus valores y explica en qué estado de agregación se encuentran a temperatura ambiente (unos 298 K).
\vspace{8pt}

\begin{tabular}{|r|c|c|c|c|}
\hline
Sustancias& $F_2$& $Cl_2$& $Br_2$& $I_2$\\
\hline
PF $(ºC)$	&-223	&-102	&-7'3		&114\\
\hline
PE $(ºC)$	&-187	&-33'7	&58'8	&182\\
\hline
\end{tabular}

\end{prob}


\begin{prob}
Señala razonadamente si las siguientes afirmaciones son verdaderas o falsas:
\begin{enumerate}
\item La fórmula del cloruro de sodio es NaCl; por tanto está formado por dos moléculas.
\item El hierro es un metal; por tanto su punto de fusión será muy alto.
\item El enlace covalente es un enlace débil; por eso el grafito (mina de los lápices) se rompe fácilmente.
\item El oxígeno forma moléculas y es un gas a temperatura ambiente. El agua forma moléculas y no es un 
gas a temperatura ambiente.
\end{enumerate}
\end{prob}


\begin{prob}
Indica el tipo de enlace esperado en cada caso para una sustancia que:
\begin{enumerate}
\item Cuando se disuelve conduce la corriente eléctrica.
\item No se puede disolver en agua ni conduce la corriente eléctrica.
\item Reacciona con el agua de forma violenta pero conduce la corriente eléctrica en estado sólido.
\item Es gaseosa a temperatura ambiente.
\end{enumerate}
\end{prob}


\end{document}