% !TEX encoding = UTF-8 Unicode
\documentclass[11pt,twoside,a4paper]{article}
\usepackage[spanish]{babel}
\usepackage{amsthm}
\usepackage[utf8]{inputenc}
\usepackage{geometry}                % See geometry.pdf to learn the layout options. There are lots.
\geometry{letterpaper}                   % ... or a4paper or a5paper or ... 
%\geometry{landscape}                % Activate for for rotated page geometry
%\usepackage[parfill]{parskip}    % Activate to begin paragraphs with an empty line rather than an indent
\usepackage{graphicx}
\usepackage[version=3]{mhchem}
\usepackage{amssymb}
\usepackage{epstopdf}
\usepackage{multicol}
\usepackage{enumerate}
\usepackage{pdfpages}

\DeclareGraphicsRule{.tif}{png}{.png}{`convert #1 `dirname #1`/`basename #1 .tif`.png}

\title{Problemas de Química}
\author{Maria Trillo Alcalá}
%\date{}                                           % Activate to display a given date or no date

%definition of the new style
\newtheoremstyle{problem}% name 
  {8pt}% Space above 
  {12pt}% Space below 
  {}% Body font
  {}% Indent amount
  {\bfseries}% Theorem head font 
  {}% Punctuation after theorem head 
  {6pt}% Space after theorem head
  {}% Theorem head spec

%declaring the style to be used
\theoremstyle{problem}
%declaring the new theorem-like structure and declaring that
%it should be numbered within chapters
\newtheorem{prob}{Ejercicio}[section]

\begin{document}

\includepdf[pages=-]{title.pdf}

\tableofcontents
\newpage

% !TEX encoding = UTF-8 Unicode
\section{Estructura atómica}

\begin{prob}
Indica toda la información contenida en \ce{^{12}_6C}; \ce{^{16}_8O^{2-}};
\ce{^{23}_{11}Na^+}.
\end{prob}

\begin{prob}
La plata natural tiene una masa atómica de 107,88. Dicho elemento tiene dos isótopos. Uno
de ellos es \ce{^{107}_{47}Ag}, el cual se encuentra en la proporción del 56\%. ¿Cuál es
la masa atómica aproximada del segundo isótopo?
\end{prob}

\begin{prob}
Si la carga del núcleo de un átomo es $3'52 \cdot 10^{-18}$C, ¿cuántos protones y 
electrones tiene? ¿Cuál es su configuración electrónica?. Dato: $e = 1'6 \cdot 10^{-19}$C
\end{prob}

\begin{prob}
El elemento carbono está formado por un $98'90\%$ de carbono-12, un $1'10\%$ de
carbono-13 y trazas de carbono-14. Calcula la masa atómica relativa del elemento carbono,
sabiendo que la masa atómica relativa del C-13 es $13'003$.
\end{prob}

\begin{prob}
La mayoría de los hornos de microondas comerciales utilizan la radiación de frecuencia
\ce{\nu = 2'45 \cdot 10^9 s^{-1}}. Calcula la longitud de onda de esta radiación.
\end{prob}

\begin{prob}
La clorofila absorbe luz azul, de 460nm de longitud de onda y emite luz roja, de 660nm de 
longitud de onda. Calcula el cambio neto de energía, en kilojulios, que experimenta la
clorofila cuando absorbe un mol de fotones de 460nm y emite un mol de fotones de 660nm.
Datos:
\begin{displaymath}
c = 3'00 \cdot 10^8 m \cdot s^{-1} \qquad
h = 6'63 \cdot 10^{-34} J \cdot s \qquad
N_A = 6'02 \cdot 10^{23} mol^{-1}
\end{displaymath}
\end{prob}

\begin{prob}
Explica cuáles de los siguientes grupos de valores correspondientes a (n,l,m) definen a
un orbital:
\begin{multicols}{3}
\begin{enumerate}[a)]
    \item $(2,0,1)$
    \item $(3,3,0)$
    \item $(1,0,1)$
    \item $(2,1,0)$
    \item $(4,3,4)$
    \item $(-5,0,1)$
\end{enumerate}
\end{multicols}
\end{prob}

\begin{prob}
Escribe las siguientes configuraciones electrónicas: \ce{_{28}Ni^{3+}}; \ce{_{19}K^{+}};
\ce{_{35}Br^-}; \ce{_{12}Sc}.
\end{prob}

\begin{prob}
Dadas las siguientes configuraciones electrónicas: \ce{(_3Li) 1s^21p^1}; 
\ce{(_5B) 1s^22s^12p^2}; \ce{(_7N) 1s^22s^22p^3}; \ce{(_8O) 1s^22s^22p^6}.
\begin{enumerate}[a)]
	\item ¿Cuáles no se corresponden con la configuración electrónica del estado 
	fundamental de los átomos que se indican?
	\item ¿Cuáles corresponden a iones?
\end{enumerate}
\end{prob}

\begin{prob}
Justifica la siguiente afirmación: Las tres configuraciones electrónicas siguientes están
relacionadas con el elemento químico de número atómico $Z = 8$.
\begin{enumerate}[a)]
	\item \ce{1s^2 2s^2 2p^4} 
	\item \ce{1s^2 2s^1 2p^5}
	\item \ce{1s^2 2s^2 2p^3}
\end{enumerate}
\end{prob}
\newpage

% !TEX encoding = UTF-8 Unicode
\section{Propiedades de la Tabla Periódica}

\begin{prob}
Indica cuántos electrones externos tienen los siguientes átomos: Ca, B, N, K, I.
Para responder es preciso considerar el grupo y el periodo de la Tabla Periódica al que pertence
cada uno de ellos. 
\end{prob}


\begin{prob}
¿Qué elementos tienen mayor radio: el bario o el calcio? ¿Cuál es el más electronegativo el O o el Ar?
\end{prob}


\begin{prob}
Dados los elementos Ar (Z=8), As (Z=33) y I (Z=53). Indica el grupo y el periodo al que pertenecen.
¿Cuál de ellos tiene una mayor energía de ionización? 
\end{prob}


\begin{prob}
¿Qué elementos tienen más tendencia a formar iones positivos, el berilio o el bario?
¿Cuál tiene más tendencia a formar iones negativos, el carbono o el oxígeno?
\end{prob}

\newpage

% !TEX encoding = UTF-8 Unicode
\section{El enlace químico}

\begin{prob}
Dibuja las estructuras de Lewis que cumplan la regla del octeto de:
\begin{enumerate}
	\item Anión carbonato
	\item Amoniaco
	\item Ácido clórico
\end{enumerate}
\end{prob}


\begin{prob}
Teniendo en cuenta la teoría de los enlaces de valencia, explica las valencias
covalentes del flúor, del oxígeno y del azufre. 
\end{prob}


\begin{prob}
Teniendo en cuenta la teoría anterior, explica y representa las siguientes moléculas:
$Cl_2$, $HCl$, $O_2$, $N_2$. ¿Cuáles presentan enlaces múltiples?
\end{prob}


\begin{prob}
¿Se precisa la misma energía para romper un enlace $\sigma$ que un $\pi$?\\
Según la T.E.V., ¿qué forma tendrá la molécula de agua? Clasifica sus enlaces covalentes
en función de su polaridad.
\end{prob}


\begin{prob}
Observa los puntos de ebullición de las siguientes sustancias. Justifica sus valores de acuerdo con
la naturaleza de sus enlaces y / o fuerzas intermoleculares.
\vspace{8pt}

\begin{tabular}{|c|c|}
\hline
Sustancia covalente& Punto de ebullición $(ºC)$\\
\hline
$O_2$&-183\\
$H_2O$&100\\
$I_2$&183\\
$C$ (diamante)&4200\\
\hline
\end{tabular}
\end{prob}


\begin{prob}
Indica la carga esperada para los iones que componen las siguientes sustancias basándote en el grupo
a que pertenecen y sus características como metales o no metales. Nombra a dichos iones.
\begin{itemize}
\item $NaCl$
\item $CaBr_2$
\item $MgS$
\item $BaO$
\item $Al_2O_3$
\item $SrH_2$
\item $KH$
\item $Na_2O_2$
\item $BnOH$
\item $NH_4F$
\end{itemize}
\end{prob}


\begin{prob}
Clasifique las siguientes sustancias según el tipo de enlace que presenten:
\begin{itemize}
\item $MgCl_2$
\item $PCl_5$
\item $Au$
\item $Fe$
\item $SO_2$
\item $H_2O$
\item $NH_3$
\end{itemize}
\end{prob}


\begin{prob}
Teniendo en cuenta los puntos de fusión y de ebullición de las siguientes sustancias que aparecen en la tabla. 
Justifica sus valores y explica en qué estado de agregación se encuentran a temperatura ambiente (unos 298 K).
\vspace{8pt}

\begin{tabular}{|r|c|c|c|c|}
\hline
Sustancias& $F_2$& $Cl_2$& $Br_2$& $I_2$\\
\hline
PF $(ºC)$	&-223	&-102	&-7'3		&114\\
\hline
PE $(ºC)$	&-187	&-33'7	&58'8	&182\\
\hline
\end{tabular}

\end{prob}


\begin{prob}
Señala razonadamente si las siguientes afirmaciones son verdaderas o falsas:
\begin{enumerate}
\item La fórmula del cloruro de sodio es NaCl; por tanto está formado por dos moléculas.
\item El hierro es un metal; por tanto su punto de fusión será muy alto.
\item El enlace covalente es un enlace débil; por eso el grafito (mina de los lápices) se rompe fácilmente.
\item El oxígeno forma moléculas y es un gas a temperatura ambiente. El agua forma moléculas y no es un 
gas a temperatura ambiente.
\end{enumerate}
\end{prob}


\begin{prob}
Indica el tipo de enlace esperado en cada caso para una sustancia que:
\begin{enumerate}
\item Cuando se disuelve conduce la corriente eléctrica.
\item No se puede disolver en agua ni conduce la corriente eléctrica.
\item Reacciona con el agua de forma violenta pero conduce la corriente eléctrica en estado sólido.
\item Es gaseosa a temperatura ambiente.
\end{enumerate}
\end{prob}
\newpage

% !TEX encoding = UTF-8 Unicode
\section{Formulación y nomenclatura de Química Inorgánica}

\begin{prob}
Nombra los siguientes compuestos:
\begin{multicols}{3}
\begin{enumerate}
	\item \ce{HBr}
	\item \ce{H_3PO_4}
	\item \ce{HIO_3}
	\item \ce{HSO_4^-}
	\item \ce{AlCl_3}
	\item \ce{Al_2(SO_4)_3}
	\item \ce{SeO_2}
	\item \ce{Pb^{2+}}
	\item \ce{N_2O_4}
	\item \ce{BaO}
	\item \ce{HgO}
	\item \ce{Cd(OH)_2}
	\item \ce{Mn_2SiO_4}
	\item \ce{Ca(HSO_4)_2}
	\item \ce{CaCO_3}
	\item \ce{ZnSO_4 7H_2O}
	\item \ce{Na_2O_2}
	\item \ce{Co_3(PO_4)_2}
	\item \ce{ZnBr_2}
	\item \ce{KMnO_4}
	\item \ce{CuI_2}
	\item \ce{Ba(NO_3)_2}
	\item \ce{CuSO_4}
	\item \ce{CoCl_2}
	\item \ce{CrCl_3}
	\item \ce{Na_2S}
	\item \ce{PO_4^{3-}}
	\item \ce{HNO_3}
	\item \ce{HCl} (aq)
	\item \ce{K_2CrO_4}
\end{enumerate}
\end{multicols}
\end{prob}

\begin{prob}
Formula los siguientes compuestos:
\begin{multicols}{2}
\begin{enumerate}
	\item Fosfato de aluminio
	\item Bromuro de amonio
	\item Ácido cianhídrico
	\item Cianuro de potasio
	\item Ión sulfito
	\item Ión yoduro
	\item Óxido de aluminio
	\item Óxido de cobre (II)
	\item Carbonato de bario
	\item Cromato de plata
	\item Catión hierro (III)
	\item Sulfuro de disodio
	\item Nitrito de sodio
	\item Ión hidrogenosulfuro
	\item Dicromato de calcio
	\item Fosfato de sodio
	\item Fosfina
	\item Bicarbonato de sodio
	\item Peróxido de hidrógeno
	\item Tiosulfato de potasio 
\end{enumerate}
\end{multicols}
\end{prob}

\newpage

% !TEX encoding = UTF-8 Unicode
\section{Disoluciones}

\begin{prob}
Calcula la masa de sulfato de cobre (II) que se necesita para preparar 250 mL de una 
disolución 1M de \ce{CuSO_4}.
\end{prob}

\begin{prob}
Calcula la concentración molar de una disolución preparada mezclando 50 mL de ácido
sulfúrico, \ce{H_2SO_4}, 0'136 M, con 70 mL de agua. Supón que los volúmenes son aditivos.
\end{prob}

\begin{prob}
Calcula la masa de carbonato de sodio, \ce{Na_2CO_3}, necesaria para preparar 1 L de una
solución al $15\%$ en masa, cuya densidad es 1'15 g / mL.
\end{prob}

\begin{prob}
Se disuelven 5 g de cloruro de hidrógeno HCl, en 35 g de agua. La densidad de la
disolución de agua. La densidad de la disolución resultante es 1'06 g / mL. Calcula la
concentración de la disolución, expresando el resultado en concentración molar, en g / L y
en porcentaje en masa.
\end{prob}

\begin{prob}
Tenemos una botella que contiene una disolución de ácido clorhídrico concentrado, de
densidad $1'175 g / cm^3$ y de $35'2\%$ de riqueza en HCl. Calcula:
\begin{enumerate}[a)]
	\item La molaridad de la disolución.
	\item El volumen de dicha disolución que se necesita para preparar 1 litro de otra
	disolución de ácido clorhídrico $0'5$ M.
\end{enumerate}
\end{prob}
\newpage

% !TEX encoding = UTF-8 Unicode
\section{Estequiometría y energía de las reacciones químicas}

\begin{prob}
Las bolsas de ''aire de seguridad'' (\textit{airbag}) de los automóviles se inflan con
nitrógeno gaseoso generado por la rápida descomposición de ácido de sodio (\ce{NaN3}):
$$ \ce{NaN3 (s) -> Na (s) + N2 (g)} $$
Si una bolsa de aire tiene un volumen de 38 L y debe llenarse con nitrógeno gaseoso a una
presión de 1'5 atm y una temperatura de $25ºC$, ¿cuántos gramos de ácido deben descomponerse?
\end{prob}

\begin{prob}
En siderurgia, la cal viva (\ce{CaO}) se combina con la sílice (\ce{SiO2}) presente en
el mineral de hierro para dar una escoria fundida de fórmula \ce{CaSiO3}.
\begin{enumerate}[a)]
	\item ¿Qué masa de escoria se obtiene a partir de una tonelada de sílice? ¿Qué masa 
	de cal viva es necesaria?
	\item La cal viva necesaria se puede obtener descomponiendo por calor la caliza 
	(\ce{CaCO3}) para dar cal y \ce{CO2}. ¿Qué masa de caliza haría falta?
\end{enumerate}
\end{prob}

\begin{prob}
Los camellos almacenan al grasa triestearina (\ce{C57H110O6}) en su giba. Además de
constituir una fuente de energía, la grasa es una fuente de agua, ya que se produce la
siguiente reacción:
$$ \ce{C57H110O6 (s) + O2 (g) -> CO2 (g) + H2O (l)} $$
¿Qué masa de agua puede obtenerse a partir de 1 kg de grasa?
\end{prob}

\begin{prob}
Escribe la reacción de combustión del pentano (\ce{C5H12}) sabiendo que se desprenden
3537 kJ / mol.
\begin{enumerate}[a)]
	\item ¿Qué cantidad de energía desprenderá una bombona de pentano de 10 kg?
	\item ¿Cuántos litros de aire (21\% de \ce{O2}) medidos en condiciones normales son 
	necesarios para su combustión?
\end{enumerate}
\end{prob}

\begin{prob}
Calcula el volumen de disolución 0'8M de ácido nítrico que reacciona con $50cm^3$ de una
disolución 2M de hidróxido de magnesio. En el proceso se obtienen nitrato de magnesio y
agua.
\end{prob}

\begin{prob}
El ácido acético, \ce{CH3COOH}, es el responsable de la acidez del vinagre. Se hacen
reaccionar 10'03 g de un vinagre comercial con hidróxido de bario $1'76 \cdot 10^{-2}M$,
siendo necesarios 137'20 mL de hidróxido para neutralizar el ácido. Calcula el porcentaje,
en masa, de ácido acético que contiene la muestra de vinagre.
\end{prob}

\begin{prob}
Se dispone de una muestra de cinc que se hace reaccionar con una disolución de ácido
clorhídrico de 1'18 g / mL de densidad y 35 \% de riqueza. Como producto de la reacción se
obtienen cloruro de cinc e hidrógeno gas. Calcula:
\begin{enumerate}[a)]
	\item La concentración molar del ácido clorhídrico.
	\item La masa de cinc de la muestra sabiendo que, para reaccionar con la misma, se
	necesitan $30cm^3$ de la disolución del ácido.
\end{enumerate}
\end{prob}

\begin{prob}
Se hacen reaccionar entre sí dos disoluciones acuosas que contienen 45 g de hidróxido de
bario y 18 g de ácido clorhídrico, respectivamente. En el proceso se forman cloruro de
bario y agua. Indica cuál es el reactivo limitante. Calcula la masa de cloruro de bario 
que se obtiene y la masa de reactivo que queda sin reaccionar.
\end{prob}

\begin{prob}
Se hacen reaccionar entre sí dos disoluciones acuosas que contienen 45 g de hidróxido de
bario y 18 g de ácido clorhídrico, respectivamente. En el proceso se forman cloruro de
bario y agua. Indica cuál es el reactivo limitante. Calcula la masa de cloruro de bario 
que se obtiene y la masa de reactivo que queda sin reaccionar.
\end{prob}

\newpage

\end{document}