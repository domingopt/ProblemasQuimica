% !TEX encoding = UTF-8 Unicode
\section{Disoluciones}

\begin{prob}
Calcula la masa de sulfato de cobre (II) que se necesita para preparar 250 mL de una 
disolución 1M de \ce{CuSO_4}.
\end{prob}

\begin{prob}
Calcula la concentración molar de una disolución preparada mezclando 50 mL de ácido
sulfúrico, \ce{H_2SO_4}, 0'136 M, con 70 mL de agua. Supón que los volúmenes son aditivos.
\end{prob}

\begin{prob}
Calcula la masa de carbonato de sodio, \ce{Na_2CO_3}, necesaria para preparar 1 L de una
solución al $15\%$ en masa, cuya densidad es 1'15 g / mL.
\end{prob}

\begin{prob}
Se disuelven 5 g de cloruro de hidrógeno HCl, en 35 g de agua. La densidad de la
disolución de agua. La densidad de la disolución resultante es 1'06 g / mL. Calcula la
concentración de la disolución, expresando el resultado en concentración molar, en g / L y
en porcentaje en masa.
\end{prob}

\begin{prob}
Tenemos una botella que contiene una disolución de ácido clorhídrico concentrado, de
densidad $1'175 g / cm^3$ y de $35'2\%$ de riqueza en HCl. Calcula:
\begin{enumerate}[a)]
	\item La molaridad de la disolución.
	\item El volumen de dicha disolución que se necesita para preparar 1 litro de otra
	disolución de ácido clorhídrico $0'5$ M.
\end{enumerate}
\end{prob}