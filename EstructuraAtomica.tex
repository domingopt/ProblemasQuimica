% !TEX encoding = UTF-8 Unicode
\section{Estructura atómica}

\begin{prob}
Indica toda la información contenida en \ce{^{12}_6C}; \ce{^{16}_8O^{2-}};
\ce{^{23}_{11}Na^+}.
\end{prob}

\begin{prob}
La plata natural tiene una masa atómica de 107,88. Dicho elemento tiene dos isótopos. Uno
de ellos es \ce{^{107}_{47}Ag}, el cual se encuentra en la proporción del 56\%. ¿Cuál es
la masa atómica aproximada del segundo isótopo?
\end{prob}

\begin{prob}
Si la carga del núcleo de un átomo es $3'52 \cdot 10^{-18}$C, ¿cuántos protones y 
electrones tiene? ¿Cuál es su configuración electrónica?. Dato: $e = 1'6 \cdot 10^{-19}$C
\end{prob}

\begin{prob}
El elemento carbono está formado por un $98'90\%$ de carbono-12, un $1'10\%$ de
carbono-13 y trazas de carbono-14. Calcula la masa atómica relativa del elemento carbono,
sabiendo que la masa atómica relativa del C-13 es $13'003$.
\end{prob}

\begin{prob}
La mayoría de los hornos de microondas comerciales utilizan la radiación de frecuencia
\ce{\nu = 2'45 \cdot 10^9 s^{-1}}. Calcula la longitud de onda de esta radiación.
\end{prob}

\begin{prob}
La clorofila absorbe luz azul, de 460nm de longitud de onda y emite luz roja, de 660nm de 
longitud de onda. Calcula el cambio neto de energía, en kilojulios, que experimenta la
clorofila cuando absorbe un mol de fotones de 460nm y emite un mol de fotones de 660nm.
Datos:
\begin{displaymath}
c = 3'00 \cdot 10^8 m \cdot s^{-1} \qquad
h = 6'63 \cdot 10^{-34} J \cdot s \qquad
N_A = 6'02 \cdot 10^{23} mol^{-1}
\end{displaymath}
\end{prob}

\begin{prob}
Explica cuáles de los siguientes grupos de valores correspondientes a (n,l,m) definen a
un orbital:
\begin{multicols}{3}
\begin{enumerate}[a)]
    \item $(2,0,1)$
    \item $(3,3,0)$
    \item $(1,0,1)$
    \item $(2,1,0)$
    \item $(4,3,4)$
    \item $(-5,0,1)$
\end{enumerate}
\end{multicols}
\end{prob}

\begin{prob}
Escribe las siguientes configuraciones electrónicas: \ce{_{28}Ni^{3+}}; \ce{_{19}K^{+}};
\ce{_{35}Br^-}; \ce{_{12}Sc}.
\end{prob}

\begin{prob}
Dadas las siguientes configuraciones electrónicas: \ce{(_3Li) 1s^21p^1}; 
\ce{(_5B) 1s^22s^12p^2}; \ce{(_7N) 1s^22s^22p^3}; \ce{(_8O) 1s^22s^22p^6}.
\begin{enumerate}[a)]
	\item ¿Cuáles no se corresponden con la configuración electrónica del estado 
	fundamental de los átomos que se indican?
	\item ¿Cuáles corresponden a iones?
\end{enumerate}
\end{prob}

\begin{prob}
Justifica la siguiente afirmación: Las tres configuraciones electrónicas siguientes están
relacionadas con el elemento químico de número atómico $Z = 8$.
\begin{enumerate}[a)]
	\item \ce{1s^2 2s^2 2p^4} 
	\item \ce{1s^2 2s^1 2p^5}
	\item \ce{1s^2 2s^2 2p^3}
\end{enumerate}
\end{prob}