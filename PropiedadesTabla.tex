% !TEX encoding = UTF-8 Unicode
\section{Tabla Periódica}

\begin{prob}
Indica cuántos electrones externos tienen los siguientes átomos: Ca, B, N, K, I.
Para responder es preciso considerar el grupo y el periodo de la Tabla Periódica al que pertence
cada uno de ellos. 
\end{prob}


\begin{prob}
¿Qué elemento tiene mayor radio: el bario o el calcio? ¿Cuál es el más electronegativo 
el O o el Ar?
\end{prob}


\begin{prob}
Dados los elementos Ar (Z=18), As (Z=33) y I (Z=53). Indica el grupo y el periodo al que pertenecen.
¿Cuál de ellos tiene una mayor energía de ionización? 
\end{prob}


\begin{prob}
¿Qué elemento tiene más tendencia a formar iones positivos, el berilio o el bario?
¿Cuál tiene más tendencia a formar iones negativos, el carbono o el oxígeno?
\end{prob}

\begin{prob}
Dado un elemento químico, A, de número atómico 35, contesta a las siguientes cuestiones:
\begin{enumerate}[a)]
\item ¿Cuántos protones y electrones tiene?
\item ¿Escribe su configuración electrónica?
\item ¿A qué grupo y a qué periodo pertenece?
\item ¿Cuál será su ión más estable?
\end{enumerate}
\end{prob}

\begin{prob}
Dadas las siguientes configuraciones electrónicas que corresponde a elementos
representativos:
\begin{multicols}{2}
\begin{enumerate}[a)]
\item $[Ne] 3s^2 3p^2$
\item $[Ne] 3s^2 3p^5$
\item $[Kr] 5s^2$
\item $[He] 2s^2 2p^5$
\end{enumerate}
\end{multicols}
Indica el número atómico de cada elemento, el periodo y el grupo al que pertenece y el
elemento del que se trata.
\end{prob}

\begin{prob}
Escribe la configuración electrónica general de:
\begin{enumerate}[a)]
\item Un metal alcalino.
\item Un halógeno.
\item Un gas noble.
\item Un ión negativo de un halógeno.
\item Un ión positivo de un metal alcalino.
\end{enumerate}
\end{prob}

\begin{prob}
La energía de ionización del ión $K^+$ es mayor que la del átomo de argón, Ar, a pesar de
que ambos poseen 18 electrones. ¿Cuál puede ser la razón?  
\end{prob}

\begin{prob}
La afinidad electrónica del flúor es $-328 kJ \cdot mol^{-1}$. Expresa este valor en 
$eV \cdot$ átomo$^{-1}$. Escribe la ecuación química en la que está implicada esta energía. 
\end{prob}

\begin{prob}
En un cristal de sodio, la distancia entre los núcleos de dos átomos contiguos es de 
372pm. Calcula el radio atómico del sodio en angstroms.
\end{prob}

