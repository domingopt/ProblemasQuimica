% !TEX encoding = UTF-8 Unicode
\section{Propiedades de la Tabla Periódica}

\begin{prob}
Indica cuántos electrones externos tienen los siguientes átomos: Ca, B, N, K, I.
Para responder es preciso considerar el grupo y el periodo de la Tabla Periódica al que pertence
cada uno de ellos. 
\end{prob}


\begin{prob}
¿Qué elementos tienen mayor radio: el bario o el calcio? ¿Cuál es el más electronegativo el O o el Ar?
\end{prob}


\begin{prob}
Dados los elementos Ar (Z=8), As (Z=33) y I (Z=53). Indica el grupo y el periodo al que pertenecen.
¿Cuál de ellos tiene una mayor energía de ionización? 
\end{prob}


\begin{prob}
¿Qué elementos tienen más tendencia a formar iones positivos, el berilio o el bario?
¿Cuál tiene más tendencia a formar iones negativos, el carbono o el oxígeno?
\end{prob}
